\documentclass{ximera}
\author{Chris Orban}
\title{You Code It}

\begin{document}
\maketitle

Click ``run'' to play around with the 1-dimensional version of planetoids.

Push \textbf{W} to see the rocket accelerate. Unfortunately right now
the ship can only accelerate in one direction. We will change this
step-by-step to make the behavior more similar to the classic
asteroids game. If the example doesn't work tell the instructor.

\begin{javascriptCode}
x = 0;
y = 0;

vx = 0;
vy = 0;

deltaVx = 0;
deltaVy = 0;

theta = 0;

Fthrust = 30.0;
mass = 3.0;
dt = 0.1;

p.draw = function() {
    // Update velocities
    vx += deltaVx;

    // Update location
    x += vx*dt;

    // velocity is unchanged if there are no forces
    deltaVx = 0;

    // Turn or thrust the ship depending on what key is pressed
    if (keyIsDown('A'.charCodeAt(0))) {
        theta += 0.0;
    }
    if (keyIsDown('D'.charCodeAt(0))) {
        theta += 0.0;
    }
    if (keyIsDown('W'.charCodeAt(0))) {
        // Rockets on!
        accelx = Fthrust*cos(theta)/mass;
        deltaVx = accelx*dt;
    }
    if (keyIsDown('S'.charCodeAt(0))) {
            // do nothing
    }
    if (keyIsPressed && key == ' '){ //spacebar
        // Do nothing!
    }
    // Draw ship and other stuff
    // This will clear the screen and re-draw it
    display();
    // Add more graphics here before the end of draw()
  
} // end draw()

////////////////////////////////////////////////////////////////

xhistory = [];
yhistory = [];

p.setup = function() {
    createCanvas(750, 500);
    x = width/2;
    y = height/2;
}

// Size of the ship
var r = 12;

var showarrows = true;

iterations = 0;

// Draw the ship and other stuff
function display() {
    wrapEdges();
    background(255);
    textSize(12);
    textStyle(NORMAL);

//    stroke(0);
    strokeWeight(10);
//    println("lets do this");
    var tri_width=7;
    if (showarrows) {
            var x_line=10;
            var y_line=25;
            var line_len=100;
            drawLine(x_line,y_line,x_line,y_line+line_len);
            drawLine(x_line,y_line,x_line+line_len,y_line);
        fill(0);
        drawTriangle(x_line-tri_width/2,y_line+line_len,x_line+tri_width/2,y_line+line_len,x_line,y_line+line_len+10);
        drawTriangle(x_line+line_len,y_line-tri_width/2,x_line+line_len,y_line+tri_width/2,x_line+line_len+10,y_line);
            strokeWeight(0);
            drawText("+x",x_line+line_len+15,y_line);
            drawText("+y",x_line,y_line+line_len+15);
    }

    if (showarrows) {
    textSize(20);
    strokeWeight(0);
    fill(255,0,0);
    text("Velocity",0.8*width,0.8*height+25);
    fill(0,0,255);
    text("Force",0.8*width,0.8*height+50);
    fill(204,0,204);
    text("Acceleration",0.8*width,0.8*height+75);
    }

    ship(x,y,vx,vy,deltaVx/dt,deltaVy/dt,theta);

  	strokeWeight(0);
    textSize(12);
    text("left right arrows to turn, tap up arrow to thrust, press H to hide the arrows, press U to un-hide",10,10);

      if (iterations%5 == 1) {  
    append(xhistory,x);
    append(yhistory,y);
    }

    iterations += 1;   
  
    if (keyIsPressed) {
     isrunning = true; 
    }
  
    MaxLength = 50;
  	if (xhistory.length > MaxLength) {
    xhistory = subset(xhistory,xhistory.length-MaxLength,xhistory.length);
    yhistory = subset(yhistory,yhistory.length-MaxLength,yhistory.length);  
    }  
  
    fill(0,0,0); //If more text is written elsewhere make sure the default is black
    stroke(0,0,0); // If more lines are drawn elsewhere make sure the default is black
    strokeWeight(0);
  
}

function wrapEdges() {
    var buffer = r*2;
    if (x > width +  buffer) x = -buffer;
    else if (x <    -buffer) x = width+buffer;
    if (y > height + buffer) y = -buffer;
    else if (y <    -buffer) y = height+buffer;
}

function drawBlob( _x,  _y){
    strokeWeight(2);
    //    fill(255);
    noFill();
    stroke(0);
    ellipse(_x, height - _y, 50, 50);  
}

function drawBlob( _x,  _y, _vx, _vy){
    strokeWeight(2);
    //    fill(255);
    noFill();
    stroke(0);
    ellipse(_x, height - _y, 50, 50);  
  
            strokeWeight(10);
    var tri_width=7;

    // Draw velocity arrow
    var v_scaling=5.0;
    stroke(255,0,0); // makes the line red
    strokeWeight(3); // makes the line thicker

    if ( ((_vx !== 0) || (_vy !== 0)) && showarrows) {
        drawLine(_x,_y,_x+v_scaling*_vx,_y+v_scaling*_vy);
        var vel_angle = -atan2(_vy,_vx);
        fill(255,0,0); // makes the triangle red
        drawTriangle(_x+v_scaling*_vx+sin(vel_angle)*tri_width/2,_y+v_scaling*_vy+cos(vel_angle)*tri_width/2,_x+v_scaling*_vx-sin(vel_angle)*tri_width/2,_y+v_scaling*_vy-cos(vel_angle)*tri_width/2,_x+v_scaling*_vx+cos(vel_angle)*10,_y+v_scaling*_vy-sin(vel_angle)*10);
    }
  


      fill(0,0,0); //If more text is written elsewhere make sure the default is black
    stroke(0,0,0); // If more lines are drawn elsewhere make sure the default is black
    strokeWeight(0);

}



function ship( _x,  _y, _vx, _vy, _ax, _ay, _theta)
{
    strokeWeight(2);
    //    fill(255);
    noFill();
    stroke(0);

    stroke(0);
    strokeWeight(2);
    push();
    translate(_x,height-_y);
    rotate(-theta+PI/2);
    fill(175);
    // A triangle
    beginShape();
    vertex(-r,r);
    vertex(0,-1.5*r);
    vertex(r,r);
    endShape(CLOSE);
    rectMode(CENTER);
    pop();
    fill(0);
    strokeWeight(0);
  
  
  
    strokeWeight(10);
    var tri_width=7;

    // Draw velocity arrow
    var v_scaling=1.0;
    stroke(255,0,0); // makes the line red
    strokeWeight(3); // makes the line thicker

    if ( ((_vx !== 0) || (_vy !== 0)) && showarrows) {
        drawLine(_x,_y,_x+v_scaling*_vx,_y+v_scaling*_vy);
        var vel_angle = -atan2(_vy,_vx);
        fill(255,0,0); // makes the triangle red
        drawTriangle(_x+v_scaling*_vx+sin(vel_angle)*tri_width/2,_y+v_scaling*_vy+cos(vel_angle)*tri_width/2,_x+v_scaling*_vx-sin(vel_angle)*tri_width/2,_y+v_scaling*_vy-cos(vel_angle)*tri_width/2,_x+v_scaling*_vx+cos(vel_angle)*10,_y+v_scaling*_vy-sin(vel_angle)*10);
    }

     // Draw force arrow
    var f_scaling=2.25;
//    var f_scaling=5.0;
    var Fx = mass*_ax;
    var Fy = mass*_ay;
    var f_angle = -atan2(Fy,Fx);

    if (((Fx !== 0) || (Fy !== 0)) && showarrows) {
//    if (((Fx != 0) || (Fy != 0)) && 0 ) {
    stroke(0,0,255); // makes the line blue
    drawLine(_x,_y,_x+f_scaling*Fx,_y+f_scaling*Fy);
    fill(0,0,255); // makes the triangle blue
    drawTriangle(_x+f_scaling*Fx+sin(f_angle)*tri_width/2,_y+f_scaling*Fy+cos(f_angle)*tri_width/2,_x+f_scaling*Fx-sin(f_angle)*tri_width/2,_y+f_scaling*Fy-cos(f_angle)*tri_width/2,_x+f_scaling*Fx+cos(f_angle)*10,_y+f_scaling*Fy-sin(f_angle)*10);    
    }
  
    var a_scaling=2.25;
    f_angle = -atan2(_ay,_ax);
    if (((_ax !== 0) || (_ay !== 0)) && showarrows) {
    stroke(204,0,204); // makes the line purple
    drawLine(_x,_y,_x+a_scaling*_ax,_y+a_scaling*_ay);
    fill(204,0,204); // makes the triangle purple
    drawTriangle(_x+a_scaling*_ax+sin(f_angle)*tri_width/2,_y+a_scaling*_ay+cos(f_angle)*tri_width/2,_x+a_scaling*_ax-sin(f_angle)*tri_width/2,_y+a_scaling*_ay-cos(f_angle)*tri_width/2,_x+a_scaling*_ax+cos(f_angle)*10,_y+a_scaling*_ay-sin(f_angle)*10);
    }
  

      fill(0,0,0); //If more text is written elsewhere make sure the default is black
    stroke(0,0,0); // If more lines are drawn elsewhere make sure the default is black
    strokeWeight(0);

}


/*
function drawBlob( _x,  _y, _r){
  strokeWeight(2);
  ellipse(_x, height - _y, _r, _r);  
}*/

function drawEllipse( _x,  _y,  _w,  _h){
  ellipse(_x, height - _y, _w, _h);  
}

function drawLine( _x1,  _y1,  _x2,  _y2){
  strokeWeight(3);
  line(_x1, height - _y1, _x2, height - _y2);  
//  strokeWeight(0);
}

function drawPoint( _x,  _y){
    strokeWeight(3);
    point(_x, height - _y);  
    strokeWeight(0);
}

function drawQuad( _x1,  _y1,  _x2,  _y2,  _x3,  _y3,  _x4,  _y4){
  quad(_x1, height - _y1, _x2, height - _y2, _x3, height - _y3, _x4, height - _y4);  
}

function drawRect( _x,  _y,  _w,  _h){
  rect(_x, height - _y, _w, _h);  
}

function drawRect( _x,  _y,  _w,  _h,  _r){
  rect(_x, height - _y, _w, _h, _r);  
}

function drawRect( _x,  _y,  _w,  _h,  _tl,  _tr,  _br,  _bl){
  rect(_x, height - _y, _w, _h, _tl, _tr, _br, _bl);  
}

function drawTriangle( _x1,  _y1,  _x2,  _y2,  _x3,  _y3){
  triangle(_x1, height - _y1, _x2, height - _y2, _x3, height - _y3);
}

function drawText( _str,  _x, _y){
    if (isNumeric(_str)){
        _str = round(100*_str)/100;
    }
    textSize(20);  
    strokeWeight(1);
    text(_str, _x, height- _y);
}

function isNumeric(n) {
    return !isNaN(parseFloat(n)) && isFinite(n);
}

\end{javascriptCode}

Look at the \textbf{draw} function used to move the ship around in the
1D version. Take a close look at the code and try to get an idea of
what each part does.

\section{Make the ship rotate}

Edit the source code above to let the ship rotate.

Change these lines of code:

\begin{verbatim}
    if (keyIsDown(LEFT_ARROW)) {
        theta += 0.0;
    } 
    if (keyIsDown(RIGHT_ARROW)) {
        theta += -0.0;
    }
\end{verbatim}

to look like this:

\begin{verbatim}
    if (keyIsDown(LEFT_ARROW)) {
        theta += 0.1;
    } 
    if (keyIsDown(RIGHT_ARROW)) {
        theta += -0.1;
    }
\end{verbatim}

The asteroids game is now set up so so that only the $x$-component of
the thrust is what accelerates the ship. Notice that allowing the ship
to rotate means that we can speed up and slow down the ship because we
can rotate the rocket and apply a force that is opposite to the
velocity vector. Our asteroids game is already a lot more fun!

\section{Enable multi-dimensional space travel}

In other words, let the ship move in the $y$ direction too!

Now take the code you've written and modify it so that the ship can
move in the y direction. It will take a few different changes to get
it to work.

\subsection{Change the ``Update location'' section}

Look at the "Update location" section. Right now it looks like this:

\begin{verbatim}
// Update location
x += vx*dt;
\end{verbatim}

As you can see, only the $x$ position is updated. The position is not
updated. Change the ``Update location'' section to this:

\begin{verbatim}
// Update location
x += vx*dt;
y += vy*dt;
\end{verbatim}

\subsection{Change the "Update velocities" section}

If you did the previous step correctly, we can now update both the $x$
position and the $y$ position, but only if $v_x$ and $v_y$ are
non-zero! This will only happen if we change the ``Update velocities''
section so that both $v_x$ and $v_y$ are updated.

Change this:

\begin{verbatim}
// Update velocities
vx += deltaVx;
\end{verbatim}

to this:
\begin{verbatim}
// Update velocities
vx += deltaVx;
vy += deltaVy;
\end{verbatim}

\subsection{Objects in motion move in a straight line unless acted on by a force}

Newton realized that objects move in a straight line with constant
velocity if there are no forces acting on the object. For our computer
program, we need to make sure that the velocity only changes when we
push buttons on the keyboard. If we don't push any buttons then the
change in velocity ($\Delta v_x$ and $\Delta v_y$) had better be zero.

This is the logic behind this part of the code:
\begin{verbatim}
    // velocity is unchanged if there are no forces
    deltaVx = 0;
\end{verbatim}
Change this to make sure deltaVy is set to zero too:
\begin{verbatim}
    // velocity is unchanged if there are no forces
    deltaVx = 0;
    deltaVy = 0;
\end{verbatim}

This is an important step because if you do the next step correctly, but you forget to do this the ship will accelerate uncontrollably.

\subsection{Make sure that acceleration works in the $y$-direction}

If you've done all the steps up to this point correctly, the ship
still won't move in the y direction because we haven't told the ship
how to accelerate in the y direction when the thrusters are turned
on. Nowhere in the program do we set deltaVy equal to anything except
zero!

Look closely at this part of the code:
\begin{verbatim}
    if (keyIsDown('W'.charCodeAt(0))) {
        // Rockets on!
        accelx = Fthrust*cos(theta)/mass;
        deltaVx = accelx*dt;
    }
\end{verbatim}
You'll need to change this to something like this:
\begin{verbatim}
    if (keyIsDown('W'.charCodeAt(0))) {
        // Rockets on!
        accelx = Fthrust*cos(theta)/mass;
        deltaVx = accelx*dt;
        accely = ????
        deltaVy = accely*dt;
    }
\end{verbatim}
It's your job to replace the ???? with something that actually gives
the correct acceleration in the $y$ direction. (Hint: it is very
similar to the code for accelx but the trig function might be
different! Should you use cosine, sine or tangent for the trig
function? If you're not sure just try one and see what happens!)

\subsection{Check that everything works as expected!}

Your ship should move in two dimensions!

Potential pitfall: Which trig function did you use to determine the $y$-component acceleration?
\begin{multipleChoice}
  \choice{cosine}
  \choice[correct]{sine}
  \choice{tangent}  
\end{multipleChoice}

\subsection{Show trajectory of the ship!}

To show the trajectory (a.k.a. path) of the ship copy this code into
the program at some point after display()
\begin{verbatim}
    for( i = 0; i < xhistory.length ; i+= 1) {
     drawPoint(xhistory[i],yhistory[i]); 
    }
\end{verbatim}

Does the path look jagged or more curvy? (If you completed the ``move
the blob'' or ``accelerate the blob'' exercises you can
compare/contrast to those trajectories.)

Comment on this in the comment box below.
\begin{freeResponse}  
\end{freeResponse}

\end{document}


